\chapter{Introduction}
\begin{chapterpage}
  \noindent
  This is the \texttt{chapterpage}\index{chapterpage} environment.
  It has to follow after every chapter, even if its contents are empty, because it sets up a few things for the remainder of the chapter.
  This environment also serves as a good place to put introductory material for the chapter, like a few paragraphs explaining what the chapter is about.

  If you put text here, it's good to put a \verb|\noindent| before the first paragraph\ldots because it's the first paragraph.

  Anything you put after \verb|\chapter| and before \texttt{chapterpage} will appear on the previous page with the big ``$CHAPTER$'' header.
  I don't recommend putting any text immediately \textit{before} the \texttt{chapterpage}; because that's what the \texttt{chapterpage} is for.
  If you do want to put something before it anyway, I recommend doing no more than an ornament or an image, and being consistent with adding one to every chapter.

  \vskip\baselineskip

  \noindent
  Surprise citation \cite{mid}!
  This is to test the bibliography (page~\pageref{chap:bibliography}) and further reading (page~\pageref{chap:furtherreading}).

  \vskip\baselineskip

  \noindent
  You can use \verb|\minitoc|\index{minitoc!control sequence} to typeset the sections and subsections of the present chapter in a format similar to the original table of contents:

  \vskip\baselineskip
  \minitoc
  \vskip\baselineskip
  \noindent
  The \textsf{minitoc}\index{minitoc!package} package does generate a ton of auxiliary files during compilation, so don't be surprised if you see the folder get filled up.%
  \footnote{\color{gray3}
  If you're unaware of what aux files are for: \LaTeX{} compiles the document in one shot, from start to finish, and doesn't backtrack.
  So if it runs into information that it needs to reference in an earlier part of the document\,---chapter and section numbers \& titles; figure numbers and captions; etc.---\, it will save that info in a file, so that the next time the compiler runs and needs this information, it can look in the relevant file to find it.}

  \vfill
  \begin{figure}[h]
    \hrulefill\vskip1pt%
    {\color{gray5}\rule{\textwidth}{\dimexpr49pt+0.4cm\relax}}%
    \vspace*{\dimexpr-47pt-0.4cm\relax}

    {\color{white}
    \ amazing figure\hfill\vskip0.2cm{\centering insane graphics\par}\vskip0.2cm\hfill love to see it\ \hbox{}\vskip1pt}\hrulefill
    \caption[Text Width Figure]{\color{gray3}Text Width Figure}
  \end{figure}
\end{chapterpage}

\noindent
The above is a \texttt{widefig},\index{widefig} one of the environments created for this document class.
It's a wrapper around the \texttt{figure} environment, so you can use \verb|\caption| inside it.
It passes ``\kern -1pt \texttt{[h]}\kern -1pt'' to the \texttt{figure} by default, but you can override that with the optional argument.

\begin{widefig}[t]%
  {\color{gray9}
  \rule{\dimexpr0.22\widefigwidth\relax}{30ex}\hfill\color{gray7}
  \rule{\dimexpr0.22\widefigwidth\relax}{30ex}\hfill\color{gray5}
  \rule{\dimexpr0.22\widefigwidth\relax}{30ex}\hfill\color{gray3}
  \rule{\dimexpr0.22\widefigwidth\relax}{30ex}% this comment is important. feel free to remove it and try to figure out what happened :)
}
  \vskip1pt
  \hrulefill
  \captionof{figure}{Extra-Width Figure}
\end{widefig}

\begin{marginfig}
  \hrulefill\vskip1pt%
  {\color{gray8}\rule{\textwidth}{\dimexpr49pt+5cm\relax}}%
  \vspace*{\dimexpr-47pt-5cm\relax}

  \ amazing figure\hfill\vskip2.5cm{\centering insane graphics\par}\vskip2.5cm\hfill love to see it\ \hbox{}\vskip1pt\hrulefill
  \captionof{figure}{Margin Figure.\label{fig:mar1}}
\end{marginfig}

To the side is another environment, the \texttt{marginfig}.\index{marginfig}
Due to some \LaTeX{} esoterica that I don't understand, the \texttt{figure} environment can't be used in margins.
What's done instead is \texttt{marginpar} with a \texttt{minipage} inside it.
So you'll need to use ``\verb|\captionof{figure}{your caption}|'' instead of ``\verb|\caption{your caption}|''.

Margin figures have no protection against flowing off the page.
To~deal with this\,---and also for times when you want to align the figure with something specific---\,you have the choice to provide to the environment an optional argument, which specifies a vertical (downward) offset.
If~you give it a negative length, the figure will go~up instead, as you might expect.
I~recommend moving margin figures by integer increments (or~decrements) of~\verb|\baselineskip|,%
\footnote{I have this set to 15\,pt.
If you're unaware, \texttt{\textbackslash{}baselineskip} is \textit{basically} a measure of the vertical distance from one~line and the next.
To clarify: look at the second paragraph of this page, beginning with ``To the side''.
The~vertical distance between the top of the `T' there, and the top of the `T' in ``\LaTeX'' on the next line, is~equal to the~\texttt{\textbackslash{}baselineskip}.}
according to how many lines you want to move the figure by.
In the case of Figure~\ref{fig:mar1}, I placed its code immediately before the ``To the side\ldots'' paragraph; but if I wanted it to start on the line containing {\setlength{\fboxsep}{0pt}\fbox{this box\strut}}, I could begin the environment as~below:
\begin{center}
  \verb|\begin{marginfig}[14\baselineskip]|
\end{center}
knowing that the box is 14 lines below where the margin figure would be placed \textit{without} the~argument.
Feel free to try it yourself: add the argument and recompile this document.

\vskip\baselineskip

\noindent
\textit{The rest of this document will serve purely as an outline for the SDP.
Do preserve the section titles and order.}

\section{Problem Statement and Purpose}
1 page
\section{Project and Design Objectives}
1 page
\section{Intended Outcomes and Deliverables}
1 page
\section{Summary of Report Structure}
2 pages
