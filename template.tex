\documentclass{aurak-sdp}

% debug
\usepackage{lipsum}
\usepackage{printlen}

% Comment out the parts you're not currently working on.
% For example, if you're doing frontmatter you can comment out
%   the entire argument, keeping only ``\includeonly{}''.
% (The way this is done, it can end up swapping right and left pages,
%   but that shouldn't happen when compiling the full report,
%   so there's no real need to fix it.)
% To compile the entire report, comment everything,
%   *including* the ``\includeonly{}''.
% Note, the bibilography, further reading, and index will only reflect
%   the parts you've kept in.
\includeonly{
  chapters/ch1/ch1,
%  chapters/ch2/ch2,
%  chapters/ch3/ch3,
%  chapters/ch4/ch4,
%  chapters/ch5/ch5,
%  chapters/ch6/ch6,
%  chapters/ch7/ch7,
%  chapters/ch8/ch8,
%  chapters/ch9/ch9,
%  raci,
%  dictionary,
%  glossary
}

\addbibresource{refs.bib}

% Replace these as necessary
\def\projtitle{A Rather Very Long Title About Some Things}
\def\shorttitle{Short Title} % For use in headers and title page. Has to be catchy.
\def\coursecode{CSCI\,492} % or whatever
\def\prof{Prof.\@ Whom Ever} % The \@ corrects the space after the dot; keep it
% Use the correct title for your supervisor!
\def\uni{American University of Ras Al Khaimah}
\def\school{School of Engineering and Computing}
\def\dept{Department of Computer Science and Engineering}
\def\yr{2024--2025}

% You may redefine document-wide colors here (in lieu of class options).
% These are the defaults; they are aliases for colors defined by the
%   `ninecolors' package. You can alter the aliases here, or outright
%   redefine the colors with `\definecolor'.
\colorlet{accentdarker}{gray3}
\colorlet{accentdark}{gray4}
\colorlet{accent}{gray7}
\colorlet{accentlight}{gray8}
\colorlet{accentlighter}{gray9}
\colorlet{titlepagecolor}{white}

% You may also renewcommand*{\maketitle} if you want,
% to rearrange the cover page. For AURAK SDPs, you
% have to keep ALL the information in there including
% the AURAK logo, but any (sane) layout will do.

% Definitely do renewcommand{\titlepagedecoration}
% and replace it with something of your own concoction.
% If you don't want one at all, uncomment this:
% \renewcommand\titlepagedecoration{}

\begin{document}
\newgeometry{inner=2.5cm,outer=2.5cm,top=3.5cm,bottom=2cm}
\pagestyle{empty}
\pdfbookmark[0]{Cover}{cover}
\fontsize{12}{15}\selectfont

\maketitle
\frontmatter % Necessary after maketitle
\studentstatement
We, the undersigned students, certify and confirm that the work submitted in this project is entirely our own and has not been copied from any other source.
Any material that has been used from other sources has been properly cited and acknowledged in the report.

We are fully aware that any copying or improper citation of references/sources used in this report will be considered plagiarism, which is a clear violation of the Code of Ethics of the \uni.

\vfill

% Insert your signatures here.
\hspace*{\fill}
\begin{minipage}{0.3\textwidth}
  \centering
  % Uncomment the below and change "sig1" to your signature's file name.
  % \includegraphics[width=\textwidth]{sig1}\\
  \hrulefill\vskip1cm Signature \vskip1cm\hrulefill\\% Placeholder. Remove this line after adding the signature image!
  % Repeat the same steps for the below.
  Name\\ID
\end{minipage}\hspace*{\fill}
\begin{minipage}{0.3\textwidth}
  \centering
  % \includegraphics[width=\textwidth]{sig2}\\
  \hrulefill\vskip1cm Signature \vskip1cm\hrulefill\\
  Name\\ID
\end{minipage}\hspace*{\fill}
\vskip2\baselineskip
\hspace*{\fill}
\begin{minipage}{0.3\textwidth}
  \centering
  % \includegraphics[width=\textwidth]{sig3}\\
  \hrulefill\vskip1cm Signature \vskip1cm\hrulefill\\
  Name\\ID
\end{minipage}\hspace*{\fill}
\begin{minipage}{0.3\textwidth}
  \centering
  % \includegraphics[width=\textwidth]{sig4}\\
  \hrulefill\vskip1cm Signature \vskip1cm\hrulefill\\
    Name\\ID
  \end{minipage}\hspace*{\fill}
\vskip2\baselineskip
\hspace*{\fill}
  % \includegraphics[width=\textwidth]{sig5}\\
\begin{minipage}{0.3\textwidth}
  \centering
  \hrulefill\vskip1cm Signature \vskip1cm\hrulefill\\
  Name\\ID
\end{minipage}\hspace*{\fill}
  % \includegraphics[width=\textwidth]{sig6}\\
\begin{minipage}{0.3\textwidth}
  \centering
  \hrulefill\vskip1cm Signature \vskip1cm\hrulefill\\
  Name\\ID
\end{minipage}\hspace*{\fill}
\vfill\vfill

\clearpage

\supervisorcertification
This is to certify that the work presented in this senior year project manuscript was carried out under my supervision, which is entitled:

\begin{center}
  \textbf{``\fulltitle''}
\end{center}

\printallstudents

\vskip\baselineskip

I hereby certify that the aforementioned students have successfully finished their senior project and by submitting this report they have fulfilled in partial the requirements of B.Sc.\@ Degrees in Computer Science, Computer Engineering, and Artificial Intelligence.

I also hereby certify that I have \textbf{read, reviewed, and corrected the technical content} of this report and I believe that it is adequate in scope, quality and content and it is in alignment with the ABET requirements and the department guidelines.

\vskip\baselineskip

\textcolor{accent}{--- \prof}

\clearpage

\acknowledgement
\textsl{ack!} Please write an acknowledgement here: to your group, your professors, your friends, your family, and God.

\clearpage\abstract
Write your abstract here.


\cleardoublepage
{
\dominitoc % Necessary before any of toc, lof, lot
\tableofcontents

\clearpage
\listoffigures

\clearpage
\listoftables

\clearpage
\listofsymbolshdr % Only the header, since the list is populated manually.
% Note this is not the same as the glossary!

\noindent
This is where you should put all the symbols used and abbreviations (must be sorted).

\begin{longtblr}[label=none,entry=none]{
    colspec={Q[r]|Q[l]}, % Fiddle with the relative widths here if needed. See tabularray package docs
    width=\textwidth
  }
  \textbf{Symbol/Abbr.} & \textbf{Meaning}\\\hline
  Abbr. & Abbreviation\\
  API & Application Programming Interface\\
  AURAK & \uni\\
  JSON & JavaScript Object Notation\\
  SDK & Software Development Kit\\
  UI & User Interface\\
  $x$ & is the symbol that marks the spot.\\
  $z$ & denotes depth (of sleep).\\
\end{longtblr}

\restoregeometry
}

\mainmatter % Necessary before chapter 1.
\chapter{Introduction}
\begin{chapterpage}
  \noindent
  This is the \texttt{chapterpage}\index{chapterpage} environment.
  It has to follow after every chapter, even if its contents are empty, because it sets up a few things for the remainder of the chapter.
  This environment also serves as a good place to put introductory material for the chapter, like a few paragraphs explaining what the chapter is about.

  If you put text here, it's good to put a \verb|\noindent| before the first paragraph\ldots because it's the first paragraph.

  Anything you put after \verb|\chapter| and before \texttt{chapterpage} will appear on the previous page with the big ``$CHAPTER$'' header.
  I don't recommend putting any text immediately \textit{before} the \texttt{chapterpage}; because that's what the \texttt{chapterpage} is for.
  If you do want to put something before it anyway, I recommend doing no more than an ornament or an image, and being consistent with adding one to every chapter.

  \vskip\baselineskip

  \noindent
  Surprise citation \cite{mid}!
  This is to test the bibliography (page~\pageref{chap:bibliography}) and further reading (page~\pageref{chap:furtherreading}).

  \vskip\baselineskip

  \noindent
  You can use \verb|\minitoc|\index{minitoc!control sequence} to typeset the sections and subsections of the present chapter in a format similar to the original table of contents:

  \vskip\baselineskip
  \minitoc
  \vskip\baselineskip
  \noindent
  The \textsf{minitoc}\index{minitoc!package} package does generate a ton of auxiliary files during compilation, so don't be surprised if you see the folder get filled up.%
  \footnote{\color{gray3}
  If you're unaware of what aux files are for: \LaTeX{} compiles the document in one shot, from start to finish, and doesn't backtrack.
  So if it runs into information that it needs to reference in an earlier part of the document\,---chapter and section numbers \& titles; figure numbers and captions; etc.---\, it will save that info in a file, so that the next time the compiler runs and needs this information, it can look in the relevant file to find it.}

  \vfill
  \begin{figure}[h]
    \hrulefill\vskip1pt%
    {\color{gray5}\rule{\textwidth}{\dimexpr49pt+0.4cm\relax}}%
    \vspace*{\dimexpr-47pt-0.4cm\relax}

    {\color{white}
    \ amazing figure\hfill\vskip0.2cm{\centering insane graphics\par}\vskip0.2cm\hfill love to see it\ \hbox{}\vskip1pt}\hrulefill
    \caption[Text Width Figure]{\color{gray3}Text Width Figure}
  \end{figure}
\end{chapterpage}

\noindent
The above is a \texttt{widefig},\index{widefig} one of the environments created for this document class.
It's a wrapper around the \texttt{figure} environment, so you can use \verb|\caption| inside it.
It passes ``\kern -1pt \texttt{[h]}\kern -1pt'' to the \texttt{figure} by default, but you can override that with the optional argument.

\begin{widefig}[t]%
  {\color{gray9}
  \rule{\dimexpr0.22\widefigwidth\relax}{30ex}\hfill\color{gray7}
  \rule{\dimexpr0.22\widefigwidth\relax}{30ex}\hfill\color{gray5}
  \rule{\dimexpr0.22\widefigwidth\relax}{30ex}\hfill\color{gray3}
  \rule{\dimexpr0.22\widefigwidth\relax}{30ex}% this comment is important. feel free to remove it and try to figure out what happened :)
}
  \vskip1pt
  \hrulefill
  \captionof{figure}{Extra-Width Figure}
\end{widefig}

\begin{marginfigure}<0\baselineskip>
  \hrulefill\vskip1pt%
  {\color{gray8}\rule{\textwidth}{\dimexpr49pt+5cm\relax}}%
  \vspace*{\dimexpr-51pt-5cm\relax}

  \ amazing figure\hfill\vskip2.5cm{\centering insane graphics\par}\vskip2.5cm\hfill love to see it\ \hbox{}\vskip-6pt\hrulefill
  \captionof{figure}{Margin Figure.\label{fig:mar1}}
\end{marginfigure}

This document class specifies the margins in such a way that the text body allows for not much more than 75~characters per line.
This leaves a lot of empty space that can be used for other content.
Placing figures in the margin is a possible use for that space; to that end, I've included the \textsf{sidenotesplus} package to help.
To the side is the \texttt{marginfigure} environment from that package.\index{marginfigure}

The \texttt{marginfigure} environment acts like a float; it places itself at what an algorithm thinks ``makes the most sense''.
If you want to make it appear exactly where you specified, you can do this:\par
{\centering \verb|\begin{marginfigure}<0pt>|\par}
The use of ``\texttt{<offset>}'', even with a zero length, forces fixed placement of the figure.

Specifying a nonzero length for \texttt{offset} will move the figure down by that length.
(Negative lengths, therefore, move the figure up.)
I~recommend moving margin figures by integer increments (or~decrements) of~\verb|\baselineskip|,%
\footnote{I have this set to 15\,pt.
  If you're unaware, \texttt{\textbackslash{}baselineskip} is a measure of the distance from one ``baseline'' to the next.
  The baseline is that on which letters ``sit''; letters like `g' and `y' go below the baseline.}
according to how many lines you want to move the figure by.
In the case of Figure~\ref{fig:mar1}, I placed its code immediately before the ``This document class\ldots'' paragraph; but if I wanted it to start on the line containing {\setlength{\fboxsep}{0pt}\fbox{this box\strut}}, I could begin the environment as~below:
\begin{center}
  \verb|\begin{marginfigure}<17\baselineskip>|
\end{center}
knowing that the box is 17 lines below where the margin figure would have been placed with \texttt{<0pt>}.
Feel free to try it yourself: add the argument and recompile this document.

\vskip\baselineskip

\noindent
\textit{The rest of this document will serve purely as an outline for the SDP.
Do preserve the section titles and order.}

\section{Problem Statement and Purpose}
1 page
\section{Project and Design Objectives}
1 page
\section{Intended Outcomes and Deliverables}
1 page
\section{Summary of Report Structure}
2 pages

\begin{chapterpage}
\end{chapterpage}
\noindent
5 pages, section it however you want

\begin{chapterpage}
\end{chapterpage}
\noindent
15 pages, section it however you want

\begin{chapterpage}
\end{chapterpage}
\section{Tasks, Schedules, and Milestones}
2 pages
\section{Application Code Management}
2 pages
\section{Resources and Cost Management}
1 page
\section{Lessons Learned}
1--2 pages

\chapter[Requirements Analysis]{Requirements Analysis}
\begin{chapterpage}
  \minitoc
\end{chapterpage}

\section{User Requirements}
1--2 pages total
\subsection{Functional User Requirements}
% TODO
\subsection{Nonfunctional User Requirements}
% TODO

\section{System Requirements}
1--2 pages total
\subsection{Functional System Requirements}
% TODO
\subsection{Nonfunctional System Requirements}
% TODO

\section{Standards \& Protocols}
\subsection{Standards}
2 pages
\subsection{Protocols}
2 pages

\begin{chapterpage}[CHAPTER~6\\SYSTEM ANALYSIS\break\& DESIGN] % TODO
\end{chapterpage}
\section{System Model}
\subsection{Sequence Diagrams}
2--4 pages
\subsection{Activity Diagrams}
1--2 pages
\subsection{Use Case Diagrams}
2--4 pages

\section{System Development}
10--20 pages

\section{User Interface Design}
5--10 pages

\section{Technical Choices}
(Discuss language, IDE, database, materials.)

\noindent
4--5 pages

\begin{chapterpage}[CHAPTER~7\\IMPLEMENTATION\break\& TESTING] % TODO
\end{chapterpage}
\section{Results}
4--5 pages
\section{Discussion}
1--2 pages

\begin{chapterpage}
\end{chapterpage}
\noindent
At least one of the following sections. 2--4 pages.
\section{Economical Impact}
\section{Societal Impact}
\section{Global Impact}
\section{Environmental Impact}
\section{Ethical Impact}

\begin{chapterpage}
\end{chapterpage}
% Section title doesn't fit into \textwidth, so break it in heading.
\section[Summary of Achievements of the Project Objectives]{Summary of Achievements of\\the Project Objectives}
1--2 pages
\section{Encountered Difficulties}
1--2 pages
\section{New Skills and Experiences Learnt}
1--2 pages
\section{Recommendations for Future Work}
1--2 pages


%% BACKMATTER
\bibchapter
\begin{chapterpage}
\end{chapterpage}\pagestyle{bib}
{
\raggedright % Looks better in my view
\nocite{*}
\printbibliography[category=cited,heading=none]
}

\frchapter % Further Reading := in bib file but not cited
\begin{chapterpage}
\end{chapterpage}\pagestyle{fr}% for real...
{
\raggedright
\printbibliography[notcategory=cited,heading=none]
}

\appendix
\setcounter{chapter}{2}
\chapter{RACI Matrix}\label{chap:raci}
\begin{chapterpage}
% you can put an explainer here for what a raci matrix is,
% or how to interpret the structure of your matrix.
% can also leave the chapterpage empty. up to you
See Appendix~\ref{chap:dictionary}.
\end{chapterpage}

\noindent
The WBS dictionary. (Format WIP)


\cleardoublepage
% ^ Has to be done manually since we're setting the counter afterward in glossary.tex.
% Otherwise, the last page of appendix D would show F--\thepage!
\setcounter{chapter}{6}
\chapter{Glossary}
\begin{chapterpage}
% short explainer? or leave the chapter page empty. up to you
\end{chapterpage}\pagestyle{gl}
\begin{description}
  \item[Term.] A very long and quite unfathomably winding definition that simply appears to have no end whatsoever.
  \item[Term.] Definition.
  \item[Term.] Definition.
  \item[Term.] Definition.
\end{description}


\cleardoublepage
% ^ same deal
\setcounter{chapter}{8}
\chapter{Index}
\begin{chapterpage}
\end{chapterpage}
\printindex
\end{document}
